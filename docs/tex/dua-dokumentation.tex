% FFHS-DUA: Datenstrukturen und Algorithmen
% Hauptdokument für ConTeXt-Kompilierung
% ========================================

% Basis-Setup
\setuppapersize[A4]
\setuplayout[
    topspace=2.5cm,
    bottomspace=2cm,
    backspace=2.5cm,
    width=16cm,
    height=24.7cm,
    header=1cm,
    footer=1cm
]

% Sprache und Encoding
\mainlanguage[de]
\enableregime[utf-8]

% Externe Style-Dateien laden
\input templates/styles/typography
\input templates/styles/colors
\input templates/styles/layout
\input templates/styles/code

% Metadaten
\setvariables[document][
    title={FFHS-DUA: Datenstrukturen und Algorithmen},
    subtitle={Praktische Vertiefungsaufgaben und Implementierungen},
    author={Daniel Senften},
    institution={Fernfachhochschule Schweiz (FFHS)},
    course={Datenstrukturen und Algorithmen},
    date={\currentdate},
    version={1.0}
]

% PDF-Metadaten
\setupinteraction[
    state=start,
    title={\getvariable{document}{title}},
    subtitle={\getvariable{document}{subtitle}},
    author={\getvariable{document}{author}},
    keyword={FFHS, Algorithmen, Datenstrukturen, Python}
]

% Hyperlinks konfigurieren
\setupurl[style=\tt, color=linkcolor]

% Kopf- und Fußzeilen werden in layout.tex definiert (inkl. Git-Info)

% Infobox Definition (für Literatur-Box)
\defineframedtext[infobox][
    width=\textwidth,
    frame=on,
    framecolor=primarycolor,
    background=color,
    backgroundcolor=lightgray,
    rulethickness=2pt,
    offset=1em,
    before={\blank[medium]},
    after={\blank[medium]}
]

% Dokument-Struktur
\starttext

% Titelseite
\startstandardmakeup
\vfill
\midaligned{\bfd\ss\color[primarycolor]{\getvariable{document}{title}}}
\blank[2*big]
\midaligned{\bfb\ss\color[secondarycolor]{\getvariable{document}{subtitle}}}
\vfill
\midaligned{\bfa\ss Kurs: \getvariable{document}{course}}
\blank[big]
\midaligned{\bf\ss \getvariable{document}{institution}}
\vfill
\midaligned{
    Autor: \getvariable{document}{author} \\
    Datum: \getvariable{document}{date} \\
    Version: \getvariable{document}{version}
}
\vfill
\stopstandardmakeup

% Inhaltsverzeichnis
\page
\midaligned{\bfd\ss\color[primarycolor]{Inhaltsverzeichnis}}
\blank[2*big]
\placecontent

% =====================
% TEIL I: GRUNDLAGEN
% =====================
\page
\input content/chapters/00_kursuebersicht

\page
\input content/chapters/01_komplexitaetsanalyse

\page
\input content/chapters/02_union_find

% =====================
% TEIL II: SORTIEREN
% =====================
\page
\input content/chapters/03_quick_sort

\page
\input content/chapters/04_heap_sort

% =====================
% ANHANG
% =====================
\page
\input content/chapters/A_erste_schritte

\stoptext
