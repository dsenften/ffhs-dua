% FFHS-DUA: Datenstrukturen und Algorithmen
% Hauptdokument für ConTeXt-Kompilierung
% ========================================

% Globale Environment-Datei laden (aus ~/Library/texmf/tex/context/user/)
\environment environment

% =============================================================================
% FFHS-SPEZIFISCHE ANPASSUNGEN
% =============================================================================

% Überschreibe primarycolor mit FFHS-Blau
\definecolor[primarycolor][ffhsblue]

% Überschreibe Kopfzeile mit FFHS-spezifischem Titel
\setupheadertexts
    [FFHS DUA - Datenstrukturen und Algorithmen][\getmarking[chapter]]

% Metadaten
\setvariables[document][
    title={FFHS-DUA: Datenstrukturen und Algorithmen},
    subtitle={Praktische Vertiefungsaufgaben und Implementierungen},
    author={Daniel Senften},
    institution={Fernfachhochschule Schweiz (FFHS)},
    course={Datenstrukturen und Algorithmen},
    date={\currentdate},
    version={1.0}
]

% PDF-Metadaten
\setupinteraction[
    state=start,
    title={\getvariable{document}{title}},
    subtitle={\getvariable{document}{subtitle}},
    author={\getvariable{document}{author}},
    keyword={FFHS, Algorithmen, Datenstrukturen, Python}
]

% Hyperlinks konfigurieren
\setupurl[style=\tt, color=linkcolor]

% Kopf- und Fußzeilen werden in layout.tex definiert (inkl. Git-Info)

% Infobox Definition (für Literatur-Box)
\defineframedtext[infobox][
    width=\textwidth,
    frame=on,
    framecolor=primarycolor,
    background=color,
    backgroundcolor=lightgray,
    rulethickness=2pt,
    offset=1em,
    before={\blank[medium]},
    after={\blank[medium]}
]

% Dokument-Struktur
\starttext

% Titelseite
\startstandardmakeup
\vfill
\midaligned{\bfd\ss\color[primarycolor]{\getvariable{document}{title}}}
\blank[2*big]
\midaligned{\bfb\ss\color[secondarycolor]{\getvariable{document}{subtitle}}}
\vfill
\midaligned{\bfa\ss Kurs: \getvariable{document}{course}}
\blank[big]
\midaligned{\bf\ss \getvariable{document}{institution}}
\vfill
\midaligned{
    Autor: \getvariable{document}{author} \\
    Datum: \getvariable{document}{date} \\
    Version: \getvariable{document}{version}
}
\vfill
\stopstandardmakeup

% Inhaltsverzeichnis
\page
\midaligned{\bfd\ss\color[primarycolor]{Inhaltsverzeichnis}}
\blank[2*big]
\placecontent

% =====================
% EINLEITUNG
% =====================
\page
\input content/chapters/00_einleitung/index
\page
\input content/chapters/00_einleitung/erste_schritte

% =====================
% MODUL 1: GRUNDLAGEN
% =====================
\page
\input content/chapters/01_grundlagen/index

% =====================
% MODUL 2: SORTIEREN
% =====================
\page
\input content/chapters/02_sortieren/index

% =====================
% MODUL 3: SUCHEN
% =====================
\page
\input content/chapters/03_suchen/index

% =====================
% MODUL 4: GRAPHEN
% =====================
\page
\input content/chapters/04_graphen/index

% =====================
% MODUL 5: STRINGS
% =====================
\page
\input content/chapters/05_strings/index

\stoptext
