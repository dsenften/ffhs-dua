% Layout-Definitionen für FFHS-DUA Dokumentation
% ===============================================

% Seitenlayout
\setuplayout[
    topspace=2.5cm,
    bottomspace=3cm,
    leftmargin=2.5cm,
    rightmargin=2.5cm,
    header=1.5cm,
    footer=2cm,
    footerdistance=0.8cm,
    width=middle,
    height=middle,
    backspace=3cm,
    cutspace=0cm
]

% Mehrspalten-Layout (optional)
\definecolumnset[twocolumn][n=2, distance=0.5cm]

% Seitenumbruch-Einstellungen (Seitenzahl wird via \setupfootertexts gesteuert)
\setuppagenumbering[
    alternative=doublesided,
    location=,
    style=\rm,
    conversion=numbers
]

% =============================================================================
% GIT-INFORMATIONEN
% =============================================================================

% Git-Branch und Commit-ID dynamisch ermitteln
\startluacode
    local function get_git_info()
        local branch = "unknown"
        local commit = "000000"

        -- Branch ermitteln
        local handle = io.popen("git rev-parse --abbrev-ref HEAD 2>/dev/null")
        if handle then
            branch = handle:read("*a"):gsub("%s+", "")
            handle:close()
        end

        -- Kurze Commit-ID ermitteln
        handle = io.popen("git rev-parse --short HEAD 2>/dev/null")
        if handle then
            commit = handle:read("*a"):gsub("%s+", "")
            handle:close()
        end

        return branch, commit
    end

    local branch, commit = get_git_info()
    document.gitbranch = branch
    document.gitcommit = commit
    document.gitinfo = "[git] • Branch: " .. branch .. "@" .. commit
\stopluacode

% Makro für Git-Info in TeX verfügbar machen
\def\gitinfo{\ctxlua{context(document.gitinfo)}}
\def\gitbranch{\ctxlua{context(document.gitbranch)}}
\def\gitcommit{\ctxlua{context(document.gitcommit)}}

% =============================================================================
% KOPF- UND FUSSZEILEN
% =============================================================================

% Markierungen für Kapitel und Section aktivieren
\setuphead[chapter][marking=chapter]
\setuphead[section][marking=section]

% Kopfzeilen-Layout - keine Linie
\setupheader[
    style=\tfx\ss,
    color=lighttext,
    before=,
    after=
]

% Kopfzeilen-Inhalt: Links = Dokumenttitel, Rechts = Kapitel
\setupheadertexts
    [FFHS DUA - Datenstrukturen und Algorithmen][\getmarking[chapter]]

% Fußzeilen-Layout - keine Linie, mehr Abstand nach oben
\setupfooter[
    style=\tfx\ss,
    color=lighttext,
    before={\blank[big]},
    after=
]

% Fußzeilen-Inhalt: Git-Info zentriert, Seitenzahl rechts
\setupfootertexts
    [][\hbox to \hsize{\hfil\gitinfo\hfil\llap{\pagenumber}}]

% Abschnittsnummerierung
\setuphead[chapter][number=yes]
\setuphead[section][number=yes]
\setuphead[subsection][number=yes]
\setuphead[subsubsection][number=yes]

% Seitenumbrüche bei Kapiteln
\setuphead[chapter][page=yes]

% Floating-Einstellungen für Abbildungen und Tabellen
\setupfloat[figure][
    location=here,
    frame=off,
    before={\blank[medium]},
    after={\blank[medium]}
]

\setupfloat[table][
    location=here,
    frame=off,
    before={\blank[medium]},
    after={\blank[medium]}
]

% Abstände zwischen Elementen
\setupblank[medium]

% Einzüge nach Überschriften
\setupindenting[no]

% Referenzierungssystem
\setupreferencing[left=, right=]

% Cross-Referenzen
\setupreferenceformat[default][
    left={},
    right={}
]

% Satzspiegel-Optimierung
\setupalign[width, tolerant, stretch, hanging]

% Trennung
\setuptolerance[verytolerant, stretch]
\setuphyphenation[method=traditional]

% Absatzeinzug nach Leerzeilen
\setupindenting[yes, medium, first]

% Keine Einzüge nach Überschriften und Listen
\setuphead[chapter, section, subsection][indentnext=no]
\setupitemize[indentnext=no]

% Marginalnoten-Position
\setupmargindata[inmargin][
    location=left,
    width=2cm,
    style=\tfx\ss\it,
    color=margincolor
]

% Seitenaufteilung für Anhänge
\definehead[appendix][chapter]
\setuphead[appendix][
    prefix=yes,
    prefixtext=Anhang,
    conversion=Character
]

% Grid-Layout (optional, für präzise Ausrichtung)
% \setupalign[line]

% Leere Seiten-Einstellungen
\setuppagenumber[state=start]
