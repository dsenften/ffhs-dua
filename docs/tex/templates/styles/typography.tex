% Typografie-Definitionen für FFHS-DUA Dokumentation
% ==================================================

% Schriftarten definieren
\definetypeface[mainface][rm][serif][palatino][default]
\definetypeface[mainface][ss][sans][helvetica][default]
\definetypeface[mainface][tt][mono][courier][default]
\definetypeface[mainface][mm][math][palatino][default]

% Hauptschriftart aktivieren
\setupbodyfont[mainface,11pt]

% Zeilenhöhe und Absatzabstände
\setupinterlinespace[line=1.2em]
\setupwhitespace[medium]

% Überschriften-Typografie
\setuphead[chapter][
    style=\bfd\ss,
    color=primarycolor,
    numberstyle=\bfd\ss,
    numbercolor=primarycolor,
    before={\blank[2*big]},
    after={\blank[big]},
    align=flushleft
]

\setuphead[section][
    style=\bfb\ss,
    color=secondarycolor,
    numberstyle=\bfb\ss,
    numbercolor=secondarycolor,
    before={\blank[big]},
    after={\blank[medium]},
    align=flushleft
]

\setuphead[subsection][
    style=\bfa\ss,
    color=accentcolor,
    numberstyle=\bfa\ss,
    numbercolor=accentcolor,
    before={\blank[medium]},
    after={\blank[small]},
    align=flushleft
]

\setuphead[subsubsection][
    style=\bf\ss,
    color=textcolor,
    numberstyle=\bf\ss,
    numbercolor=textcolor,
    before={\blank[small]},
    after={\blank[small]},
    align=flushleft
]

% Absatz-Formatierung
\setupindenting[yes, medium]
\setupalign[width, tolerant, stretch]

% Hervorhebungen
\definehighlight[important][style=\bf, color=primarycolor]
\definehighlight[emphasis][style=\it]
\definehighlight[code][style=\tt, color=codecolor]
\definehighlight[filename][style=\tt\bf, color=filenamecolor]

% Fußnoten
\setupfootnotes[
    location=page,
    rule=on,
    width=\textwidth,
    bodyfont=9pt,
    style=\rm
]

% Marginalnoten
\setupmargindata[inmargin][
    style=\tfx\ss,
    color=margincolor,
    align=flushleft
]

% Beschriftungen (Captions)
\setupcaption[figure][
    style=\tfx\ss,
    color=captioncolor,
    align=middle,
    location=bottom
]

\setupcaption[table][
    style=\tfx\ss,
    color=captioncolor,
    align=middle,
    location=top
]

% Spezielle Textblöcke
\defineframedtext[infobox][
    frame=on,
    framecolor=infocolor,
    background=color,
    backgroundcolor=infobgcolor,
    corner=round,
    radius=3pt,
    offset=10pt,
    style=\rm,
    before={\blank[medium]},
    after={\blank[medium]}
]

\defineframedtext[warningbox][
    frame=on,
    framecolor=warningcolor,
    background=color,
    backgroundcolor=warningbgcolor,
    corner=round,
    radius=3pt,
    offset=10pt,
    style=\rm,
    before={\blank[medium]},
    after={\blank[medium]}
]

\defineframedtext[tipbox][
    frame=on,
    framecolor=tipcolor,
    background=color,
    backgroundcolor=tipbgcolor,
    corner=round,
    radius=3pt,
    offset=10pt,
    style=\rm,
    before={\blank[medium]},
    after={\blank[medium]}
]

% Mathematik-Typografie
\setupmathematics[
    autopunctuation=yes,
    integral=nolimits
]

% Formelnummerierung
\setupformulas[
    location=right,
    left={(},
    right={)},
    numberstyle=\rm
]

% Tabellen-Typografie
\setupTABLE[each][each][
    style=\rm,
    align=middle,
    offset=5pt
]

\setupTABLE[row][first][
    style=\bf\ss,
    color=white,
    background=color,
    backgroundcolor=primarycolor
]

% Listen-Typografie
\setupitemize[1][
    symbol=•,
    margin=1em,
    distance=0.5em,
    style=\rm
]

\setupitemize[2][
    symbol=◦,
    margin=1.5em,
    distance=0.5em,
    style=\rm
]

\setupitemize[3][
    symbol=▪,
    margin=2em,
    distance=0.5em,
    style=\rm
]

% Nummerierte Listen
\setupenumerate[1][
    conversion=numbers,
    margin=1em,
    distance=0.5em,
    style=\rm
]

% Beschreibungslisten
\setupdescription[
    headstyle=\bf\ss,
    headcolor=primarycolor,
    margin=1em,
    distance=0.5em,
    style=\rm
]
