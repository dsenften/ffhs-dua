% Code-Highlighting und Syntax-Definitionen
% =========================================

% Basis-Setup für Code-Blöcke
\setuptyping[
    style=\tt,
    color=codecolor,
    before={\blank[medium]},
    after={\blank[medium]},
    margin=1em,
    numbering=line,
    numbercolor=lighttext,
    numberstyle=\tfx,
    frame=on,
    framecolor=mediumgray,
    background=color,
    backgroundcolor=codebgcolor,
    corner=round,
    radius=3pt,
    offset=10pt
]

% Python Syntax-Highlighting
\definetyping[python]
\setuptyping[python][
    option=python,
    tab=4,
    style=\tt\tfx
]

% Java Syntax-Highlighting
\definetyping[java]
\setuptyping[java][
    option=java,
    tab=4,
    style=\tt\tfx
]

% Shell/Bash Syntax-Highlighting
\definetyping[shell]
\setuptyping[shell][
    option=shell,
    style=\tt\tfx,
    color=shellcommand
]

% SQL Syntax-Highlighting
\definetyping[sql]
\setuptyping[sql][
    option=sql,
    style=\tt\tfx
]

% XML/HTML Syntax-Highlighting
\definetyping[xml]
\setuptyping[xml][
    option=xml,
    style=\tt\tfx
]

% JSON Syntax-Highlighting
\definetyping[json]
\setuptyping[json][
    option=json,
    style=\tt\tfx
]

% YAML Syntax-Highlighting
\definetyping[yaml]
\setuptyping[yaml][
    option=yaml,
    style=\tt\tfx
]

% Inline-Code
\definetype[code][
    style=\tt,
    color=codecolor,
    before=,
    after=
]

% Dateinamen
\definetype[filename][
    style=\tt\bf,
    color=filenamecolor,
    before=,
    after=
]

% Pfade
\definetype[path][
    style=\tt,
    color=pathcolor,
    before=,
    after=
]

% Kommandozeilen-Befehle
\definetype[command][
    style=\tt\bf,
    color=shellcommand,
    before=,
    after=
]

% Algorithmus-Pseudocode
\definetyping[algorithm]
\setuptyping[algorithm][
    style=\tt,
    numbering=line,
    numbercolor=lighttext,
    numberstyle=\tfx,
    frame=on,
    framecolor=primarycolor,
    background=color,
    backgroundcolor=lightgray,
    corner=round,
    radius=3pt,
    offset=10pt,
    before={\blank[medium]},
    after={\blank[medium]}
]

% Code-Blöcke mit Titel
\defineframedtext[codeblock][
    frame=on,
    framecolor=mediumgray,
    background=color,
    backgroundcolor=codebgcolor,
    corner=round,
    radius=3pt,
    offset=10pt,
    before={\blank[medium]},
    after={\blank[medium]}
]

% Spezielle Code-Umgebungen
\defineframedtext[pythoncode][
    frame=on,
    framecolor=pythonkeyword,
    background=color,
    backgroundcolor=codebgcolor,
    corner=round,
    radius=3pt,
    offset=10pt,
    before={\blank[medium]},
    after={\blank[medium]}
]

\defineframedtext[javacode][
    frame=on,
    framecolor=javakeyword,
    background=color,
    backgroundcolor=codebgcolor,
    corner=round,
    radius=3pt,
    offset=10pt,
    before={\blank[medium]},
    after={\blank[medium]}
]

% Terminal/Console Output
\definetyping[terminal]
\setuptyping[terminal][
    style=\tt\tfx,
    color=white,
    background=color,
    backgroundcolor=textcolor,
    frame=on,
    framecolor=textcolor,
    corner=round,
    radius=3pt,
    offset=10pt,
    before={\blank[medium]},
    after={\blank[medium]}
]

% Diff/Patch Darstellung
\definetyping[diff]
\setuptyping[diff][
    style=\tt\tfx,
    frame=on,
    framecolor=mediumgray,
    background=color,
    backgroundcolor=lightgray,
    corner=round,
    radius=3pt,
    offset=10pt,
    before={\blank[medium]},
    after={\blank[medium]}
]

% Konfigurationsdateien
\definetyping[config]
\setuptyping[config][
    style=\tt\tfx,
    frame=on,
    framecolor=accentcolor,
    background=color,
    backgroundcolor=codebgcolor,
    corner=round,
    radius=3pt,
    offset=10pt,
    before={\blank[medium]},
    after={\blank[medium]}
]

% Output/Ergebnis-Blöcke
\defineframedtext[output][
    frame=on,
    framecolor=successcolor,
    background=color,
    backgroundcolor=infobgcolor,
    corner=round,
    radius=3pt,
    offset=10pt,
    style=\tt\tfx,
    before={\blank[medium]},
    after={\blank[medium]}
]

% Fehler-Output
\defineframedtext[error][
    frame=on,
    framecolor=errorcolor,
    background=color,
    backgroundcolor=errorbgcolor,
    corner=round,
    radius=3pt,
    offset=10pt,
    style=\tt\tfx,
    before={\blank[medium]},
    after={\blank[medium]}
]
