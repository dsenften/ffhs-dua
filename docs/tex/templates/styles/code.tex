% Code-Highlighting und Syntax-Definitionen
% =========================================

% =============================================================================
% CODE-BLÖCKE MIT SYNTAX-HIGHLIGHTING (via vim)
% =============================================================================

% vim-Modul laden für Syntax-Highlighting
\usemodule[vim]

% Python-Code mit Syntax-Highlighting
\definevimtyping[pythoncode][
    syntax=python,
    before={\blank[small]\startframedtext[
        frame=on,
        framecolor=codeborder,
        background=color,
        backgroundcolor=codebackground,
        width=\textwidth,
        align=flushleft,
        loffset=0.5em,
        roffset=0.5em,
        toffset=0.3em,
        boffset=0.3em,
    ]},
    after={\stopframedtext\blank[small]},
    style=\tt\small,
    numbering=no,
]

% Bash/Shell-Code mit Syntax-Highlighting
\definevimtyping[bashcode][
    syntax=sh,
    before={\blank[small]\startframedtext[
        frame=on,
        framecolor=codeborder,
        background=color,
        backgroundcolor=codebackground,
        width=\textwidth,
        align=flushleft,
        loffset=0.5em,
        roffset=0.5em,
        toffset=0.3em,
        boffset=0.3em,
    ]},
    after={\stopframedtext\blank[small]},
    style=\tt\small,
    numbering=no,
]

% JSON-Code mit Syntax-Highlighting
\definevimtyping[jsoncode][
    syntax=json,
    before={\blank[small]\startframedtext[
        frame=on,
        framecolor=codeborder,
        background=color,
        backgroundcolor=codebackground,
        width=\textwidth,
        align=flushleft,
        loffset=0.5em,
        roffset=0.5em,
        toffset=0.3em,
        boffset=0.3em,
    ]},
    after={\stopframedtext\blank[small]},
    style=\tt\small,
    numbering=no,
]

% YAML-Code mit Syntax-Highlighting
\definevimtyping[yamlcode][
    syntax=yaml,
    before={\blank[small]\startframedtext[
        frame=on,
        framecolor=codeborder,
        background=color,
        backgroundcolor=codebackground,
        width=\textwidth,
        align=flushleft,
        loffset=0.5em,
        roffset=0.5em,
        toffset=0.3em,
        boffset=0.3em,
    ]},
    after={\stopframedtext\blank[small]},
    style=\tt\small,
    numbering=no,
]

% Generischer Text-Code (ohne Highlighting)
\definetyping[textcode][
    before={\blank[small]\startframedtext[
        frame=on,
        framecolor=codeborder,
        background=color,
        backgroundcolor=codebackground,
        width=\textwidth,
        align=flushleft,
        loffset=0.5em,
        roffset=0.5em,
        toffset=0.3em,
        boffset=0.3em,
    ]},
    after={\stopframedtext\blank[small]},
    style=\tt\small,
]

% Inline-Code
\definetype[code][style=\tt\small]

% Inline Python (mit Highlighting)
\definevimtyping[inlinepython][syntax=python,escape=on]

% =============================================================================
% ZUSÄTZLICHE CODE-UMGEBUNGEN
% =============================================================================

% Dateinamen
\definetype[filename][
    style=\tt\bf,
    color=filenamecolor,
    before=,
    after=
]

% Pfade
\definetype[path][
    style=\tt,
    color=pathcolor,
    before=,
    after=
]

% Kommandozeilen-Befehle
\definetype[command][
    style=\tt\bf,
    color=shellcommand,
    before=,
    after=
]

% Terminal/Console Output
\definetyping[terminal]
\setuptyping[terminal][
    style=\tt\tfx,
    color=white,
    background=color,
    backgroundcolor=textcolor,
    frame=on,
    framecolor=textcolor,
    corner=round,
    radius=3pt,
    offset=10pt,
    before={\blank[medium]},
    after={\blank[medium]}
]

% Output/Ergebnis-Blöcke
\defineframedtext[output][
    frame=on,
    framecolor=successcolor,
    background=color,
    backgroundcolor=infobgcolor,
    corner=round,
    radius=3pt,
    offset=10pt,
    style=\tt\tfx,
    before={\blank[medium]},
    after={\blank[medium]}
]

% Fehler-Output
\defineframedtext[error][
    frame=on,
    framecolor=errorcolor,
    background=color,
    backgroundcolor=errorbgcolor,
    corner=round,
    radius=3pt,
    offset=10pt,
    style=\tt\tfx,
    before={\blank[medium]},
    after={\blank[medium]}
]
