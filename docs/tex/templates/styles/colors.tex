% Farbdefinitionen für FFHS-DUA Dokumentation
% ============================================

% FFHS Corporate Colors
\definecolor[ffhsblue][r=0.0, g=0.3, b=0.6]      % FFHS Hauptblau
\definecolor[ffhsred][r=0.8, g=0.1, b=0.1]       % FFHS Akzentrot
\definecolor[ffhsgray][r=0.4, g=0.4, b=0.4]      % FFHS Grau

% Primäre Farbpalette
\definecolor[primarycolor][ffhsblue]               % Hauptfarbe
\definecolor[secondarycolor][ffhsgray]             % Sekundärfarbe
\definecolor[accentcolor][ffhsred]                 % Akzentfarbe

% Text-Farben
\definecolor[textcolor][r=0.1, g=0.1, b=0.1]      % Haupttext (fast schwarz)
\definecolor[lighttext][r=0.4, g=0.4, b=0.4]      % Heller Text
\definecolor[linkcolor][r=0.0, g=0.2, b=0.8]      % Hyperlinks

% Hintergrund-Farben
\definecolor[backgroundcolor][r=1.0, g=1.0, b=1.0] % Weiß
\definecolor[lightgray][r=0.95, g=0.95, b=0.95]   % Sehr helles Grau
\definecolor[mediumgray][r=0.85, g=0.85, b=0.85]  % Mittleres Grau

% Code-Highlighting Farben
\definecolor[codecolor][r=0.2, g=0.2, b=0.2]      % Code-Text
\definecolor[codebgcolor][r=0.97, g=0.97, b=0.97] % Code-Hintergrund
\definecolor[keywordcolor][r=0.0, g=0.0, b=0.8]   % Schlüsselwörter
\definecolor[stringcolor][r=0.0, g=0.6, b=0.0]    % Strings
\definecolor[commentcolor][r=0.6, g=0.6, b=0.6]   % Kommentare
\definecolor[numbercolor][r=0.8, g=0.4, b=0.0]    % Zahlen

% Syntax-Highlighting für verschiedene Sprachen
% Python
\definecolor[pythonkeyword][r=0.0, g=0.5, b=0.0]  % Python Keywords
\definecolor[pythonstring][r=0.7, g=0.1, b=0.1]   % Python Strings
\definecolor[pythoncomment][r=0.4, g=0.4, b=0.4]  % Python Kommentare

% Java
\definecolor[javakeyword][r=0.5, g=0.0, b=0.5]    % Java Keywords
\definecolor[javastring][r=0.0, g=0.0, b=0.8]     % Java Strings
\definecolor[javacomment][r=0.0, g=0.5, b=0.0]    % Java Kommentare

% Shell/Bash
\definecolor[shellcommand][r=0.0, g=0.0, b=0.0]   % Shell Commands
\definecolor[shellparam][r=0.5, g=0.0, b=0.5]     % Parameter

% Dateinamen und Pfade
\definecolor[filenamecolor][r=0.3, g=0.3, b=0.7]  % Dateinamen
\definecolor[pathcolor][r=0.4, g=0.4, b=0.4]      % Pfade

% Spezielle Textblöcke
% Info-Boxen
\definecolor[infocolor][r=0.0, g=0.4, b=0.8]      % Info-Rahmen
\definecolor[infobgcolor][r=0.9, g=0.95, b=1.0]   % Info-Hintergrund

% Warnung-Boxen
\definecolor[warningcolor][r=0.9, g=0.6, b=0.0]   % Warnung-Rahmen
\definecolor[warningbgcolor][r=1.0, g=0.98, b=0.9] % Warnung-Hintergrund

% Tipp-Boxen
\definecolor[tipcolor][r=0.0, g=0.7, b=0.0]       % Tipp-Rahmen
\definecolor[tipbgcolor][r=0.9, g=1.0, b=0.9]     % Tipp-Hintergrund

% Fehler-Boxen
\definecolor[errorcolor][r=0.8, g=0.0, b=0.0]     % Fehler-Rahmen
\definecolor[errorbgcolor][r=1.0, g=0.9, b=0.9]   % Fehler-Hintergrund

% Tabellen-Farben
\definecolor[tableheader][primarycolor]            % Tabellenkopf
\definecolor[tableodd][r=1.0, g=1.0, b=1.0]      % Ungerade Zeilen
\definecolor[tableeven][r=0.98, g=0.98, b=0.98]  % Gerade Zeilen

% Diagramm-Farben
\definecolor[diagramblue][r=0.2, g=0.4, b=0.8]    % Diagramm Blau
\definecolor[diagramgreen][r=0.2, g=0.7, b=0.2]   % Diagramm Grün
\definecolor[diagramred][r=0.8, g=0.2, b=0.2]     % Diagramm Rot
\definecolor[diagramorange][r=0.9, g=0.5, b=0.1]  % Diagramm Orange
\definecolor[diagrampurple][r=0.6, g=0.2, b=0.8]  % Diagramm Lila

% Mathematik-Farben
\definecolor[mathcolor][r=0.0, g=0.0, b=0.0]      % Mathematische Formeln
\definecolor[mathnumber][r=0.0, g=0.0, b=0.8]     % Formelnummern

% Margin und Caption Farben
\definecolor[margincolor][r=0.5, g=0.5, b=0.5]    % Marginalnoten
\definecolor[captioncolor][r=0.3, g=0.3, b=0.3]   % Bildunterschriften

% Algorithmus-Farben (für Pseudocode)
\definecolor[algokeyword][r=0.0, g=0.0, b=0.8]    % Algorithmus Keywords
\definecolor[algocomment][r=0.0, g=0.6, b=0.0]    % Algorithmus Kommentare
\definecolor[algoline][r=0.6, g=0.6, b=0.6]       % Zeilennummern

% Performance-Indikatoren
\definecolor[fastcolor][r=0.0, g=0.8, b=0.0]      % Schnell (grün)
\definecolor[mediumcolor][r=0.9, g=0.6, b=0.0]    % Mittel (orange)
\definecolor[slowcolor][r=0.8, g=0.0, b=0.0]      % Langsam (rot)

% Komplexitäts-Farben
\definecolor[bigoconstant][r=0.0, g=0.6, b=0.0]   % O(1) - Konstant
\definecolor[bigolog][r=0.0, g=0.4, b=0.8]        % O(log n) - Logarithmisch
\definecolor[bigolinear][r=0.9, g=0.6, b=0.0]     % O(n) - Linear
\definecolor[bigonlogn][r=0.8, g=0.4, b=0.0]      % O(n log n)
\definecolor[bigoquadratic][r=0.8, g=0.0, b=0.0]  % O(n²) - Quadratisch
\definecolor[bigoexponential][r=0.6, g=0.0, b=0.6] % O(2^n) - Exponentiell

% Datenstruktur-Farben
\definecolor[arraycolor][r=0.2, g=0.6, b=0.8]     % Arrays
\definecolor[listcolor][r=0.8, g=0.4, b=0.2]      % Listen
\definecolor[treecolor][r=0.2, g=0.8, b=0.2]      % Bäume
\definecolor[graphcolor][r=0.8, g=0.2, b=0.8]     % Graphen
\definecolor[hashcolor][r=0.6, g=0.6, b=0.2]      % Hash-Tabellen

% Status-Farben
\definecolor[successcolor][r=0.0, g=0.7, b=0.0]   % Erfolg
\definecolor[pendingcolor][r=0.9, g=0.6, b=0.0]   % Ausstehend
\definecolor[failurecolor][r=0.8, g=0.0, b=0.0]   % Fehler

% Transparenz-Definitionen (für Overlays)
\definecolor[lighttransparent][r=0.9, g=0.9, b=0.9, a=1, t=0.3]
\definecolor[mediumtransparent][r=0.5, g=0.5, b=0.5, a=1, t=0.5]
