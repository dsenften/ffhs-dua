% Modul 1: Grundlagen
% ===================

\part{Modul 1: Grundlagen}

\chapter{Einführung in Modul 1: Grundlagen}

\section{Überblick}

Das erste Modul legt die Fundamente für das Verständnis von Algorithmen und
Datenstrukturen. Es behandelt die grundlegenden Konzepte, die für alle
weiteren Module essentiell sind.

\subsection{Lernziele}

\startlernziele
{\bf Lernziele -- Nach Abschluss dieses Moduls können Sie:}
\startitemize
\item erklären, was Algorithmen sind und deren Eigenschaften benennen
\item Rekursionen entwerfen und damit Probleme lösen
\item den Begriff ADT (Abstrakter Datentyp) erklären und implementieren
\item die grundlegenden Datenstrukturen (Stack, Queue, Bag) benennen und anwenden
\item mit Python Algorithmen schreiben und testen
\item die Laufzeit von Algorithmen analysieren und beurteilen
\stopitemize
\stoplernziele

\subsection{Themen}

Dieses Modul umfasst folgende Hauptthemen:

\startitemize
\item {\bf Komplexitätsanalyse:} O-Notation und asymptotische Analyse
\item {\bf Union-Find:} Dynamische Konnektivität und disjunkte Mengen
\item {\bf Fundamentale Datenstrukturen:} Stack, Queue, Bag
\stopitemize

\subsection{Referenzen im Lehrbuch}

Die Themen dieses Moduls werden in folgenden Kapiteln des Lehrbuchs
{\em Praktische Algorithmik mit Python} (Häberlein, 2012) behandelt:

\startitemize
\item Kapitel 1: Algorithmen-Grundlagen
\item Kapitel 2: Rekursion und Komplexität
\item Kapitel 3: Abstrakte Datentypen
\stopitemize

% Unterkapitel einbinden
\input content/chapters/01_grundlagen/komplexitaetsanalyse
\input content/chapters/01_grundlagen/union_find
