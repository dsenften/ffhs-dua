% Modul 4: Graphen
% =================

\part{Modul 4: Graphen}

\chapter{Einführung in Modul 4: Graphen}

\section{Überblick}

Das vierte Modul behandelt Graphentheorie und Algorithmen auf Graphen,
die in vielen praktischen Anwendungen essentiell sind. Von Netzwerk-Routing
über soziale Netzwerke bis hin zu Navigationssystemen -- Graphenalgorithmen
sind überall im Einsatz.

\subsection{Lernziele}

\startlernziele
{\bf Lernziele -- Nach Abschluss dieses Moduls können Sie:}
\startitemize
\item das Konzept der Graphen und deren Bedeutung nennen
\item Breiten- und Tiefensuche erklären und anwenden
\item kürzeste Wege in einem Graphen berechnen
\item minimale Spannbäume ermitteln
\item den maximalen Fluss in einem Graphen ermitteln
\stopitemize
\stoplernziele

\subsection{Geplante Themen}

Dieses Modul wird folgende Hauptthemen umfassen:

\startitemize
\item {\bf Graph-Repräsentation:} Adjazenzlisten und Adjazenzmatrizen
\item {\bf Breitensuche (BFS):} Schichtweise Exploration
\item {\bf Tiefensuche (DFS):} Exploration in die Tiefe
\item {\bf Dijkstra's Algorithmus:} Kürzeste Wege mit nicht-negativen Gewichten
\item {\bf Minimum Spanning Tree:} Kruskal's und Prim's Algorithmus
\item {\bf Maximum Flow:} Ford-Fulkerson Algorithmus
\stopitemize

\subsection{Referenzen im Lehrbuch}

Die Themen dieses Moduls werden in folgenden Kapiteln des Lehrbuchs
{\em Praktische Algorithmik mit Python} (Häberlein, 2012) behandelt:

\startitemize
\item Kapitel 9: Graphentheorie
\item Kapitel 10: Graphenalgorithmen
\item Kapitel 11: Kürzeste Wege
\item Kapitel 12: Minimale Spannbäume
\stopitemize

\startframedtext[width=\textwidth, frame=on, framecolor=orange]
{\bf Hinweis:} Die detaillierten Kapitel zu diesem Modul werden in einer
späteren Version dieses Skripts ergänzt.
\stopframedtext
