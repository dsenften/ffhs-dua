\section{Tourverschmelzung beim TSP und Clarke-Wright-Savings-Algorithmus}

\subsection{Zusammenhang zwischen Tourverschmelzung und Clarke-Wright}

Ja, die Tourverschmelzung hat {\bf direkten Zusammenhang} mit dem Clarke-Wright-Savings-Algorithmus. Beide Verfahren basieren auf dem gleichen Grundprinzip:

\startitemize[n]
\item {\bf Ausgangssituation}: Mehrere separate (Teil-)Touren
\item {\bf Verschmelzung}: Systematisches Zusammenführen dieser Touren
\item {\bf Optimierungskriterium}: Maximale Einsparung (Savings) bei der Verschmelzung
\stopitemize

\subsection{Der Clarke-Wright-Savings-Algorithmus in einfachen Schritten}

\subsubsection{Schritt 1: Initialzustand - Sternkonfiguration}

{\bf Was passiert:}

\startitemize
\item Wir starten mit einem zentralen Depot (oft Knoten 0)
\item Jeder Kunde wird einzeln vom Depot aus besucht und wieder zurück
\item Wir haben also n separate "Touren" der Form: Depot → Kunde i → Depot
\stopitemize

{\bf Beispiel:}

\starttyping
Depot (0) → Kunde 1 → Depot (0)
Depot (0) → Kunde 2 → Depot (0)
Depot (0) → Kunde 3 → Depot (0)
Depot (0) → Kunde 4 → Depot (0)
\stoptyping

\subsubsection{Schritt 2: Savings-Berechnung}

{\bf Was passiert:}

\startitemize
\item Für jedes Kundenpaar (i, j) berechnen wir die potenzielle Einsparung
\item Die Savings-Formel lautet: {\bf s(i,j) = d(0,i) + d(0,j) - d(i,j)}
\stopitemize

{\bf Bedeutung:}

\startitemize
\item d(0,i) = Distanz vom Depot zu Kunde i
\item d(0,j) = Distanz vom Depot zu Kunde j
\item d(i,j) = Direkte Distanz zwischen Kunde i und j
\item s(i,j) = Eingesparte Distanz bei Verbindung von i und j
\stopitemize

{\bf Intuition:}
Statt zwei separate Touren zu fahren (0→i→0 und 0→j→0), fahren wir eine kombinierte Tour (0→i→j→0). Die Einsparung ist: (0→i→0) + (0→j→0) - (0→i→j→0)

\subsubsection{Schritt 3: Sortierung der Savings}

{\bf Was passiert:}

\startitemize
\item Alle berechneten Savings-Werte werden in {\bf absteigender Reihenfolge} sortiert
\item Wir erstellen eine Savings-Liste: s(i,j) ≥ s(k,l) ≥ s(m,n) ≥ ...
\stopitemize

{\bf Beispiel:}

\starttyping
s(1,3) = 25  (höchste Einsparung)
s(2,4) = 18
s(1,2) = 12
s(3,4) = 8
s(2,3) = 5   (niedrigste Einsparung)
\stoptyping

\subsubsection{Schritt 4: Tourverschmelzung (iterativ)}

{\bf Was passiert:}

\startitemize
\item Wir gehen die sortierte Savings-Liste von oben nach unten durch
\item Für jedes Kundenpaar (i,j) prüfen wir: Können wir die Touren verschmelzen?
\stopitemize

{\bf Verschmelzungsbedingungen:}

\startitemize[n]
\item Kunde i ist {\bf Endpunkt} einer Tour (nicht in der Mitte)
\item Kunde j ist {\bf Endpunkt} einer anderen Tour (nicht in der Mitte)
\item Beide Kunden sind in {\bf verschiedenen} Touren
\item Die Kapazitätsbeschränkung wird nicht verletzt (falls vorhanden)
\stopitemize

{\bf Wenn alle Bedingungen erfüllt sind:}
→ Verbinde die beiden Touren über die Kante (i,j)

\subsubsection{Schritt 5: Terminierung}

{\bf Was passiert:}

\startitemize
\item Der Algorithmus endet, wenn alle Savings abgearbeitet sind
\item Resultat: Eine oder mehrere Touren (je nach Kapazitätsbeschränkungen)
\stopitemize

\subsection{Detailliertes Beispiel mit Visualisierung}

\subsubsection{Ausgangssituation}

Gegeben: 4 Kunden (1,2,3,4) und 1 Depot (0)

{\bf Distanzmatrix:}

\placetable[here][tab:distance-matrix]{Distanzmatrix für Clarke-Wright-Beispiel}
\starttabulate[|c|c|c|c|c|c|]
\HL
\NC \NC {\bf 0} \NC {\bf 1} \NC {\bf 2} \NC {\bf 3} \NC {\bf 4} \NC\NR
\HL
\NC {\bf 0} \NC 0 \NC 10 \NC 15 \NC 20 \NC 25 \NC\NR
\NC {\bf 1} \NC 10 \NC 0 \NC 35 \NC 30 \NC 20 \NC\NR
\NC {\bf 2} \NC 15 \NC 35 \NC 0 \NC 15 \NC 30 \NC\NR
\NC {\bf 3} \NC 20 \NC 30 \NC 15 \NC 0 \NC 15 \NC\NR
\NC {\bf 4} \NC 25 \NC 20 \NC 30 \NC 15 \NC 0 \NC\NR
\HL
\stoptabulate

\subsubsection{Schritt-für-Schritt Durchführung}

{\bf 1. Initialzustand (Sternkonfiguration)}

\starttyping
Tour 1: 0 → 1 → 0  (Länge: 20)
Tour 2: 0 → 2 → 0  (Länge: 30)
Tour 3: 0 → 3 → 0  (Länge: 40)
Tour 4: 0 → 4 → 0  (Länge: 50)

Gesamtlänge: 140
\stoptyping

{\bf 2. Savings berechnen}

Für alle Paare (i,j):

\starttyping
s(1,2) = d(0,1) + d(0,2) - d(1,2) = 10 + 15 - 35 = -10 (negativ!)
s(1,3) = d(0,1) + d(0,3) - d(1,3) = 10 + 20 - 30 = 0
s(1,4) = d(0,1) + d(0,4) - d(1,4) = 10 + 25 - 20 = 15  ✓
s(2,3) = d(0,2) + d(0,3) - d(2,3) = 15 + 20 - 15 = 20  ✓
s(2,4) = d(0,2) + d(0,4) - d(2,4) = 15 + 25 - 30 = 10  ✓
s(3,4) = d(0,3) + d(0,4) - d(3,4) = 20 + 25 - 15 = 30  ✓ (max!)
\stoptyping

{\bf 3. Sortierte Savings-Liste}

\placetable[here][tab:sorted-savings]{Sortierte Savings-Liste}
\starttabulate[|l|l|]
\HL
\NC {\bf Rang} \NC {\bf Savings} \NC\NR
\HL
\NC 1. \NC s(3,4) = 30 \NC\NR
\NC 2. \NC s(2,3) = 20 \NC\NR
\NC 3. \NC s(1,4) = 15 \NC\NR
\NC 4. \NC s(2,4) = 10 \NC\NR
\NC 5. \NC s(1,3) = 0 \NC\NR
\NC 6. \NC s(1,2) = -10 \NC\NR
\HL
\stoptabulate

{\bf 4. Verschmelzung durchführen}

{\bf Iteration 1: s(3,4) = 30}

\startitemize
\item Kunde 3 ist Endpunkt von Tour 3: 0→3→0 ✓
\item Kunde 4 ist Endpunkt von Tour 4: 0→4→0 ✓
\item Verschiedene Touren ✓
\item {\bf Verschmelzung möglich!}
\stopitemize

\starttyping
Neue Tour: 0 → 3 → 4 → 0  (Länge: 20 + 15 + 25 = 60)
Vorher: (0→3→0) + (0→4→0) = 40 + 50 = 90
Ersparnis: 90 - 60 = 30 ✓

Aktuelle Touren:
Tour 1: 0 → 1 → 0  (20)
Tour 2: 0 → 2 → 0  (30)
Tour 3/4: 0 → 3 → 4 → 0  (60)
Gesamtlänge: 110 (Ersparnis: 30)
\stoptyping

{\bf Iteration 2: s(2,3) = 20}

\startitemize
\item Kunde 2 ist Endpunkt von Tour 2: 0→2→0 ✓
\item Kunde 3 ist Endpunkt von Tour 3/4: 0→3→4→0 ✓
\item Verschiedene Touren ✓
\item {\bf Verschmelzung möglich!}
\stopitemize

\starttyping
Neue Tour: 0 → 2 → 3 → 4 → 0  (Länge: 15 + 15 + 15 + 25 = 70)
Vorher: (0→2→0) + (0→3→4→0) = 30 + 60 = 90
Ersparnis: 90 - 70 = 20 ✓

Aktuelle Touren:
Tour 1: 0 → 1 → 0  (20)
Tour 2/3/4: 0 → 2 → 3 → 4 → 0  (70)
Gesamtlänge: 90 (Ersparnis: 50)
\stoptyping

{\bf Iteration 3: s(1,4) = 15}

\startitemize
\item Kunde 1 ist Endpunkt von Tour 1: 0→1→0 ✓
\item Kunde 4 ist Endpunkt von Tour 2/3/4: 0→2→3→4→0 ✓
\item Verschiedene Touren ✓
\item {\bf Verschmelzung möglich!}
\stopitemize

\starttyping
Neue Tour: 0 → 2 → 3 → 4 → 1 → 0
Länge: 15 + 15 + 15 + 20 + 10 = 75

Aktuelle Touren:
Tour gesamt: 0 → 2 → 3 → 4 → 1 → 0  (75)
Gesamtlänge: 75 (Ersparnis: 65)
\stoptyping

\subsubsection{Endergebnis}

\startitemize
\item {\bf Optimierte Tour:} 0 → 2 → 3 → 4 → 1 → 0
\item {\bf Länge:} 75
\item {\bf Ersparnis gegenüber Initialzustand:} 140 - 75 = 65 (46\% Reduktion!)
\stopitemize

\subsection{Wichtige Hinweise für die Lehre}

\subsubsection{1. Negative Savings}

\startitemize
\item Wenn s(i,j) < 0: Die Verschmelzung würde die Tour {\bf verlängern}
\item Solche Paare werden {\bf nicht} verschmolzen
\item Beispiel: s(1,2) = -10 bedeutet, es ist besser, separate Touren zu behalten
\stopitemize

\subsubsection{2. Reihenfolge ist wichtig}

\startitemize
\item Wir bearbeiten Savings in {\bf absteigender} Reihenfolge
\item Grund: Maximale Einsparungen zuerst realisieren
\item Greedy-Ansatz: Lokale Optima führen zu guten (nicht optimalen) Lösungen
\stopitemize

\subsubsection{3. Endpunkt-Bedingung}

\startitemize
\item Nur Endpunkte können verschmolzen werden
\item Warum? Sonst entstehen ungültige Touren (kein Hamiltonscher Kreis)
\item Beispiel: In 0→2→3→0 ist Kunde 3 ein Endpunkt, Kunde 2 nicht
\stopitemize

\subsubsection{4. Kapazitätsbeschränkungen (CVRP)}

\startitemize
\item Bei Vehicle Routing Problems mit Kapazität
\item Zusätzliche Bedingung: Gesamtbedarf der Tour ≤ Fahrzeugkapazität
\item Falls verletzt: Verschmelzung ablehnen
\stopitemize

\subsection{Pädagogische Tipps}

\startitemize[n]
\item {\bf Visualisierung}: Zeichnen Sie jeden Schritt auf der Tafel/Folien
\item {\bf Hands-On}: Lassen Sie Studierende ein kleines Beispiel selbst durchrechnen
\item {\bf Intuition}: Betonen Sie die Savings-Intuition (Wegfall von 2× Depot-Kanten)
\item {\bf Komplexität}: O(n²) für Savings-Berechnung, O(n² log n) für Sortierung
\item {\bf Qualität}: Typischerweise 5-15\% über optimaler Lösung (sehr gut für Heuristik!)
\stopitemize

\subsection{Vergleich zu anderen Heuristiken}

\placetable[here][tab:heuristics-comparison]{Vergleich verschiedener TSP-Heuristiken}
\starttabulate[|l|l|l|l|]
\HL
\NC {\bf Heuristik} \NC {\bf Konstruktionsart} \NC {\bf Qualität} \NC {\bf Komplexität} \NC\NR
\HL
\NC Clarke-Wright \NC Verschmelzung \NC Gut (5-15\%) \NC O(n² log n) \NC\NR
\NC Nearest Neighbor \NC Einfügung \NC Mittel (20-30\%) \NC O(n²) \NC\NR
\NC 2-Opt \NC Verbesserung \NC Sehr gut \NC O(n²) pro Iteration \NC\NR
\HL
\stoptabulate

\subsection{Zusammenfassung}

Der Clarke-Wright-Savings-Algorithmus ist eine {\bf konstruktive Heuristik}, die durch systematische {\bf Tourverschmelzung} arbeitet:

\startitemize[n]
\item Start: Sternkonfiguration (n einzelne Touren)
\item Berechne alle möglichen Einsparungen (Savings)
\item Sortiere Savings absteigend
\item Verschmelze Touren greedily nach Savings-Wert
\item Beachte: Nur Endpunkte, verschiedene Touren, Kapazität
\stopitemize

{\bf Kernidee der Tourverschmelzung:}
Zwei Depot-Kanten fallen weg, eine Kunden-Kante kommt hinzu → Netto-Einsparung!

\blank[medium]

{\it Hinweis: Diese Zusammenfassung wurde für die praktische Anwendung im FFHS-DUA-Kurs erstellt und dient als Lehrmaterial für Graphen-Algorithmen.}
