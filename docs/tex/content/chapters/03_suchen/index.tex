% Modul 3: Suchen
% ================

\part{Modul 3: Suchen}

\chapter{Einführung in Modul 3: Suchen}

\section{Überblick}

Das dritte Modul behandelt Suchalgorithmen und Datenstrukturen für
effizientes Suchen und Speichern von Schlüssel-Wert-Paaren. Sie lernen
verschiedene Symbol-Table-Implementierungen kennen, von einfachen binären
Suchbäumen bis hin zu selbstbalancierenden Varianten und Hashtabellen.

\subsection{Lernziele}

\startlernziele
{\bf Lernziele -- Nach Abschluss dieses Moduls können Sie:}

\startitemize
\item die Herausforderung der Suche benennen
\item die Funktionsweise binärer Suchbäume erklären und implementieren
\item den Unterschied zu den verschiedenen balancierten Suchbäumen aufzeigen
\item Hashtabellen für die Suche anwenden
\item die verschiedenen Suchen zweckmässig in Problemstellungen anwenden
\item den Bloomfilter Algorithmus erklären und anwenden
\stopitemize
\stoplernziele

\subsection{Geplante Themen}

Dieses Modul wird folgende Hauptthemen umfassen:

\startitemize
\item {\bf Binary Search Tree (BST):} Unbalancierte binäre Suchbäume
\item {\bf AVL Tree:} Selbstbalancierende Bäume mit Rotationen
\item {\bf Red-Black BST:} Left-Leaning Red-Black Trees
\item {\bf Hash Tables:} Separate Chaining und Linear Probing
\item {\bf Bloomfilter:} Probabilistische Datenstruktur
\stopitemize

\subsection{Referenzen im Lehrbuch}

Die Themen dieses Moduls werden in folgenden Kapiteln des Lehrbuchs
{\em Praktische Algorithmik mit Python} (Häberlein, 2012) behandelt:

\startitemize
\item Kapitel 6: Suchen und Symbol Tables
\item Kapitel 7: Balancierte Suchbäume
\item Kapitel 8: Hashtabellen
\stopitemize

\startframedtext[width=\textwidth, frame=on, framecolor=orange]
{\bf Hinweis:} Die detaillierten Kapitel zu diesem Modul werden in einer
späteren Version dieses Skripts ergänzt.
\stopframedtext
