% Modul 5: Strings und komplexe Probleme
% =======================================

\part{Modul 5: Strings und komplexe Probleme}

\chapter{Einführung in Modul 5: Strings und komplexe Probleme}

\section{Überblick}

Das fünfte Modul behandelt Algorithmen für die Verarbeitung von
Zeichenketten (Strings) sowie komplexe algorithmische Probleme und
Lösungsstrategien. Sie lernen effiziente String-Suchalgorithmen,
Trie-Datenstrukturen und fortgeschrittene Problemlösungstechniken kennen.

\subsection{Lernziele}

\startlernziele
{\bf Lernziele -- Nach Abschluss dieses Moduls können Sie:}
\startitemize
\item mittels Tries Algorithmen Zeichenketten in einem Text suchen
\item die verschiedenen Stringmatching Algorithmen erklären und anwenden
\item das Travelling-Salesman-Problem erklären und Lösungsstrategien benennen
\item den Greedy-Algorithmus erklären und anwenden
\item das Konzept der dynamischen Programmierung erklären
\stopitemize
\stoplernziele

\subsection{Geplante Themen}

Dieses Modul wird folgende Hauptthemen umfassen:

\startitemize
\item {\bf Trie:} Präfixbaum für String-Schlüssel
\item {\bf Patricia-Trie:} Komprimierte Trie-Variante
\item {\bf Knuth-Morris-Pratt (KMP):} Lineare String-Suche mit DFA
\item {\bf Boyer-Moore:} Sublineare String-Suche
\item {\bf Rabin-Karp:} Hash-basierte String-Suche
\item {\bf Greedy-Algorithmen:} Lokale Optima für globale Lösungen
\item {\bf Dynamische Programmierung:} Optimal Substructure
\item {\bf NP-vollständige Probleme:} Travelling Salesman, etc.
\stopitemize

\subsection{Referenzen im Lehrbuch}

Die Themen dieses Moduls werden in folgenden Kapiteln des Lehrbuchs
{\em Praktische Algorithmik mit Python} (Häberlein, 2012) behandelt:

\startitemize
\item Kapitel 13: String-Algorithmen
\item Kapitel 14: Greedy-Algorithmen
\item Kapitel 15: Dynamische Programmierung
\item Kapitel 16: NP-vollständige Probleme
\stopitemize

\startframedtext[width=\textwidth, frame=on, framecolor=orange]
{\bf Hinweis:} Die detaillierten Kapitel zu diesem Modul werden in einer
späteren Version dieses Skripts ergänzt.
\stopframedtext
